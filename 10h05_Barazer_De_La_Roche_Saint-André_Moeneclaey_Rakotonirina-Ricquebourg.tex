\documentclass[c,12pt]{beamer}
 \usepackage[T1]{fontenc} 
 \usepackage[latin1]{inputenc}
 \usepackage[french]{babel}
 \selectlanguage{french}
 \usetheme{Warsaw}
 \usepackage{pslatex}
 \usepackage{graphics}
\usepackage{amsfonts}
\usepackage{graphicx}
\usepackage{stmaryrd}
\usepackage{nccmath}
\usepackage{subfigure}

\newcommand{\A}[0]{{\cal A}}
\newcommand{\C}[0]{{\cal C}}
\newcommand{\I}[0]{\cal I}
\newcommand{\se}[1]{\medbreak \medbreak \section*{#1}}
\newcommand{\sse}[1]{\medbreak \subsection*{#1}}
\newcommand{\ssse}[1]{\subsubsection*{#1}}
\newcommand{\Def}[0]{\ssse{Définition}}
\newcommand{\The}[0]{\ssse{Théorème}}
\newcommand{\Pro}[0]{\ssse{Propriété}}
\newcommand{\mb}[1]{\mathbb{#1}}
\newcommand{\Exo}[1]{\sse{Exercice #1}}
\newcommand{\Que}[1]{\ssse{Question #1}}
\newcommand{\ssi}[0]{\Leftrightarrow}
\newcommand{\ra}[0]{\rightarrow}
\newcommand{\Ra}[0]{\Rightarrow}
\newcommand{\la}[0]{\leftarrow}
\newcommand{\La}[0]{\Leftarrow}
\newcommand{\llb}[0]{\llbracket}
\newcommand{\rrb}[0]{\rrbracket}
\newcommand{\dd}[0]{\txt{d}}
\newcommand{\dr}[0]{\partial}
\newcommand{\txt}[1]{\textrm{#1}}
\newcommand{\ol}[1]{\overline{#1}}
\newcommand{\ul}[1]{\underline{#1}}
\newcommand{\abs}[1]{|#1|}
\newcommand{\iso}[0]{\simeq}
\renewcommand{\P}[0]{\cal P}
\newcommand{\Fo}[0]{{\cal F}_0}
\newcommand{\w}[0]{\omega}
\newcommand{\mo}[0]{\models}
\newcommand{\blo}[2]{\begin{block}{#1} #2 \end{block}}
\newcommand{\cols}[1]{\begin{columns}#1\end{columns}}
\newcommand{\col}[2]{\begin{column}{#1}#2\end{column}}
\newcommand{\fig}[3]{\begin{figure} \includegraphics[scale = #1]{#2}\caption{#3}\end{figure}}
\newcommand{\image}[2]{\begin{figure} \includegraphics[scale = #1]{#2}\end{figure}}


\newcommand{\matrice}[1]{\left(\begin{matrix} #1 \end{matrix}\right)}
\newcommand{\subfig}[1]{\subfigure{\includegraphics[width=40mm]{#1}}}
\newcommand{\Arrow}{{ \raisebox{10\height}{\scalebox{1}{$\longrightarrow$}}}}
\newcommand{\arrow}{{\raisebox{15\height}{\scalebox{1}{$\longrightarrow$}}}}

\newcommand{\fram}[2]{\begin{frame} \frametitle{#1} #2 \end{frame}}

%\usepackage[french,onelanguage]{algorithm2e}
%\usepackage{algorithmique}

\title{Échantillonnage irrégulier pour une homographie}
\subtitle{Encadré par Jean-Michel Morel et Enric Meinhardt-Llopis}
 \author{Simon Barazer, Jean-Thomas De La Roche Saint-André, Hugo Moeneclaey, Shmuel Rakotonirina--Ricquebourg}
 \date{30 Juin 2015}

 \AtBeginSection[]{
  \begin{frame}{Sommaire}
  \small \tableofcontents[currentsection, hideothersubsections]
  {\scriptsize   
  
  Bibliographie :
     
 \  \underline{Pyramidal parametrics} L. Williams - 1983
  
   \underline{Texture mapping polygons in perspective} PS. Heckbert - 1983
 
   \underline{Real-time rendering} T. Akenine-Möller, E. Haines, N. Hoffman - 2008 
    
   \underline{High-quality multi-pass image resampling} R. Szeliski, S. Winder, \& M. Uyttendaele - 2010
    
    }
  \end{frame} 
}

 \begin{document}
 
\maketitle

 %introduction by shmuel
  \section{Qu'est-ce qu'une homographie ?}
  \begin{frame}
  \frametitle{Position du problème}
   \small{Formule générale : $H : (x,y)\mapsto \left(\frac{ax+by+p}{rx+sy+t},\frac{cx+dy+q}{rx+sy+t}\right)$ telle que $img_f(x,y)=img(H(x,y))$}
   \begin{figure}
    \centering
    \subfig{BriquesOriginal.png}
    \arrow
    \subfig{BriquesTransformed.png}
    \caption{Effet d'une homographie}
   \end{figure}
  \end{frame}
  \begin{frame}
    \frametitle{Position du problème}
   \small{Formule générale : $H : (x,y)\mapsto \left(\frac{ax+by+p}{rx+sy+t},\frac{cx+dy+q}{rx+sy+t}\right)$ telle que $img_f(x,y)=img(H(x,y))$}
   \begin{figure}
    \centering
    \subfig{BriquesOriginal.png}
    \arrow
    \subfig{BriquesTransformedExtended.png}
    \caption{Effet d'une homographie prolongée périodiquement}
   \end{figure}
  \end{frame}



 
 \fram{Méthode naïve}{
 	\cols{
 		\col{5cm}{$img_f(x,y)=img(H(x,y))$ 
			\medbreak
			On applique naïvement cette formule : valeur du pixel le plus proche
			\fig{0.25}{imageproque.jpg}{Pourquoi cela ne fonctionne pas}
		}
		\col{5cm}{
			\fig{0.3}{barbara}{Une application naïve sur une image périodique}
		}
 	}
 }
 %Et slide sur la zone à intégrer
 
 
 %moi+jeanthomas
 \section{Méthodes actuelles}
 
 \subsection{Mipmap}
 
\fram{Approximation à l'aide du Mipmap}{
	\cols{
		\col{5cm}{
			Présenté par Williams, \underline{Pyramidal parametrics}, 1983 %le mettre en bas de page ?
 %$\ra$ Précalcul de zones%eventuellemnt image des parallélogramme
			\image{0.3}{approx.jpg}
			\blo{Problèmes}
			{
				\begin{itemize}
				\item Fonctions de distance
				\item Construction du Mipmap
				\end{itemize}
			}
		}
		\col{5cm}{\fig{0.35}{MipMap_real}{Un exemple de Mipmap}}
	}
}
 
 
 \fram{L'interpolation trilinéaire}{\image{0.5}{intertri.jpg}}
 
 
 \fram{}{\fig{0.4}{Image_naive_homo_1}{Déformation de l'image par la méthode naïve}}
 
 \fram{}{\fig{0.4}{Image_mipmap_homo_1}{Déformation à l'aide d'un Mipmap avec filtre gaussien}}
 
 \subsection{Ripmap}
 
 \fram{Le Ripmap}{
	\cols{ 
 		\col{5cm}{L'isotropie n'est pas satisfaisante. 
			\image{0.2}{carrepasbien.jpg}
			$\ra$ introduction d'anisotropie 
			%blablabla
			\medbreak
			Fonctions de distance dans $\mb{R}^2$
		}
		\col{5cm}{\fig{0.25}{Ripmap_real}{Un exemple de Ripmap}}
 	}
 }
 
  \fram{L'interpolation quadrilinéaire}{\image{0.35}{interbibi.jpg}}
  
  \fram{}{\fig{0.4}{Image_mipmap_homo_2}{Déformation de l'image à l'aide d'un Mipmap}}
   
  \fram{}{\fig{0.4}{Image_ripmap_homo_2}{Déformation de l'image à l'aide d'un Ripmap}}
     
     
     
    
     
     
     
     
     
     
     
     
     
     
 \section{Directions à explorer}

%truc simon a inséréer sur affine

\subsection{Méthode anti-aliasing pour les affinités}

\begin{frame}{Etude des \emph{shears} (cisaillement)}

Introduit dans Szeliski, \underline{High quality multi-pass image resampling} en 2010. %mettre en bas de page ?

\begin{itemize}
\item minimiser l'aliasing sans filtrage préalable.
\end{itemize}

\begin{columns}
\begin{column}{5cm}

\begin{block}{Algorithme pour un \emph{shear}}

$\ra$ trois étapes :

\begin{itemize}
\item zoom avant
\item \emph{shear}
\item filtrage passe bas et zoom arrière
\end{itemize}

\end{block}
  
 \end{column}

\begin{column}{5cm}

\begin{figure}
\centering
\includegraphics[width=4cm]{dragonshear.jpg}
\caption{\emph{Shear} horizontal}
\end{figure}

\end{column}
\end{columns}
\end{frame}

\begin{frame}{Illustration de la méthode dans le domaine de Fourier }

\begin{figure}
\centering
\includegraphics[width=7cm]{fourier1.jpg}
\caption{Méthode de Szeliski}
\end{figure}

\begin{columns}
\begin{column}{4cm}

\begin{figure}
\centering
\includegraphics[width=3cm]{fourier2.jpg}
\caption{Méthode naïve}
\end{figure}

%On observe un repliement du spectre ce qui provoque de l'aliasing.

\end{column}

\begin{column}{6cm}

\begin{figure}
\centering
\includegraphics[width=5cm]{fourier3.jpg}
\caption{Méthode par filtrage}
\end{figure}

%Une partie du spectre a été perdue lors du filtrage

\end{column}
\end{columns}

\end{frame}

\begin{frame}{Méthode finale pour les affinités}
%rajouter qqvh....
\begin{itemize}
\item Affinités $\ra$ composée de deux \emph{shears}.
%$A=S_v \circ S_h$
\item translations sur les lignes et les colonnes.
%\item On réutilise la méthode développée pour les \emph{shears}.
%\item L'algorithme final se décompose en 6 étapes que l'on peut réduire à  4.
\end{itemize}

\begin{block}{Schémas de l'algorithme final}

\begin{itemize}
\item Zoom avant vertical
\item \emph{Shear} horizontal
\item \emph{Shear} vertical
\item Filtrage passe bas et zoom arrière horizontal
\end{itemize}

\end{block}

\end{frame}


%eventuellement expérience affinité

\fram{}{\fig{0.4}{shearAliase.png}{\emph{Shear} par la méthode naïve}}

\fram{}{\fig{0.4}{shearPasAliase.png}{\emph{Shear} par la méthode de Szeliski}}

\begin{frame}
\begin{figure}
\centering
\includegraphics[width=6cm]{zoom_shear_aliase.png}
\caption{Méthode naïve}
\end{figure}
\end{frame}

\begin{frame}
\begin{figure}
\centering
\includegraphics[width=6cm]{zoom_shear_pas_aliase.png}
\caption{Méthode de Szeliski}
\end{figure}
\end{frame}


 \subsection{Décomposition d'une homographie}

  \subsubsection{Homographie unidirectionnelle}

  \begin{frame} %ici tout est à changer
  \frametitle{Décomposition d'une homographie}
   Se ramener à  $H_{N,\lambda} : (x,y) \mapsto \left(\frac{\lambda x}{1+Nx},\frac{\lambda y}{1+Nx}\right)$ par des affinités :
   \[H = A_1 H_{N,\lambda} A_2\]
   \begin{figure}
    \centering
  \subfig{balloon}
    \arrow
    \subfig{Image_homo_particuliere_opt}
    \caption{Effet de l'homographie particulière}
   \end{figure}
  \end{frame}
  

  
\subsubsection{Visualisation d'une décomposition}
  \begin{frame}
  Cas de la décomposition $H=A_1H_{N,\lambda}A_2$ avec $A_1$ une affinité et $A_2$ une rotation.
  \begin{figure}
   \subfigure[$img(x,y)$]{\includegraphics[width=40mm]{VisuDecompo1.jpg}}
   \Arrow
   \subfigure[$img(A_1(x,y))$]{\includegraphics[width=40mm]{VisuDecompo2.jpg}}
   \caption{Première affinité}
  \end{figure}
  \end{frame}
  
  \begin{frame}
  Cas de la décomposition $H=A_1H_{N,\lambda}A_2$ avec $A_1$ une affinité et $A_2$ une rotation.
  \begin{figure}
   \subfigure[$img(A_1(x,y))$]{\includegraphics[width=40mm]{VisuDecompo2.jpg}}
   \caption{Première image intermédiaire}
  \end{figure}
  \end{frame}
  
  \begin{frame}
  Cas de la décomposition $H=A_1H_{N,\lambda}A_2$ avec $A_1$ une affinité et $A_2$ une rotation.
  \begin{figure}
   \subfigure[$img(A_1(x,y))$]{\includegraphics[width=40mm]{VisuDecompo2pti.jpg}}
   \Arrow
   \subfigure[$img(A_1H_{N,\lambda}(x,y))$]{\includegraphics[width=40mm]{VisuDecompo3pti.jpg}}
   \caption{Homographie particulière}
  \end{figure}
  \end{frame}
  
  \begin{frame}
  Cas de la décomposition $H=A_1H_{N,\lambda}A_2$ avec $A_1$ une affinité et $A_2$ une rotation.
  \begin{figure}
   \subfigure[$img(A_1(x,y))$]{\includegraphics[width=40mm]{VisuDecompo2grid.jpg}}
   \Arrow
   \subfigure[$img(A_1H_{N,\lambda}(x,y))$]{\includegraphics[width=40mm]{VisuDecompo3grid.jpg}}
   \caption{Homographie particulière}
  \end{figure}
  \end{frame}
  
  \begin{frame}
  Cas de la décomposition $H=A_1H_{N,\lambda}A_2$ avec $A_1$ une affinité et $A_2$ une rotation.
  \begin{figure}
   \subfigure[$img(A_1(x,y))$]{\includegraphics[width=40mm]{VisuDecompo2pti.jpg}}
   \Arrow
   \subfigure[$img(A_1H_{N,\lambda}(x,y))$]{\includegraphics[width=40mm]{VisuDecompo3pti.jpg}}
   \caption{Homographie particulière}
  \end{figure}
  \end{frame}
  
  \begin{frame}
  Cas de la décomposition $H=A_1H_{N,\lambda}A_2$ avec $A_1$ une affinité et $A_2$ une rotation.
  \begin{figure}
   \subfigure[$img(A_1H_{N,\lambda}(x,y))$]{\includegraphics[width=40mm]{VisuDecompo3.jpg}}
   \caption{Deuxième image intermédiaire}
  \end{figure}
  \end{frame}
  
  \begin{frame}
  Cas de la décomposition $H=A_1H_{N,\lambda}A_2$ avec $A_1$ une affinité et $A_2$ une rotation.
  \begin{figure}
   \subfigure[$img(A_1H_{N,\lambda}(x,y))$]{\includegraphics[width=40mm]{VisuDecompo3.jpg}}
   \Arrow
   \subfigure[$img(A_1H_{N,\lambda}A_2(x,y))$]{\includegraphics[width=40mm]{VisuDecompo4.jpg}}
   \caption{Rotation}
  \end{figure}
  \end{frame}
  
  \begin{frame}
  Cas de la décomposition $H=A_1H_{N,\lambda}A_2$ avec $A_1$ une affinité et $A_2$ une rotation.
  \begin{figure}
   \subfigure[$img(x,y)$]{\includegraphics[width=30mm]{VisuDecompo1.jpg}}
   \raisebox{7\height}{$\longrightarrow$}
   \subfigure[$img(A_1(x,y))$]{\includegraphics[width=30mm]{VisuDecompo2.jpg}}\\
   $\downarrow$
   \phantom{$\longrightarrow$}
   \phantom{$\longrightarrow$}
   \phantom{$\longrightarrow$}
   \phantom{$\longrightarrow$}
   \phantom{$\longrightarrow$}
   $\downarrow$\\
   \subfigure[$img(H(x,y))$]{\includegraphics[width=30mm]{VisuDecompo4.jpg}}
   \raisebox{7\height}{\scalebox{1}{$\longleftarrow$}}
   \subfigure[$img(A_1H_{N,\lambda}(x,y))$]{\includegraphics[width=30mm]{VisuDecompo3.jpg}}
   \caption{Homographie totale}
  \end{figure}
  \end{frame}






\begin{frame}{Perspectives}

\textbf{A suivre :} Travail sur $H_{N,\lambda}$ :\\
	\begin{itemize}
                \item décomposition lignes-colonnes ?
                \item à l'aide d'une méthode similaire à celle des affinités ?
                \item à l'aide d'une méthode intégrale ?
                \item à l'aide d'un mip-map unidimensionnel ? 
                \item interprétation géométrique par caméra ?
                \end{itemize}
% je ne sais pas quel nom donner à sa méthode...
\begin{tabular}{c}
		\includegraphics[width=0.5\textwidth]{murhomographie.jpg}\\
                       {\small Ex : Méthode par séparation}
	\end{tabular}
\end{frame}






  \appendix
  \section{Appendice}
\fram{}{  $(x,y)\mapsto \left(\frac{ax+by+p}{rx+sy+t},\frac{cx+dy+q}{rx+sy+t}\right)$ est notée $\matrice{a&b&p\\c&d&q\\r&s&t}$. %cette frame doit être allégrement supprimée
   
   Si $H = \matrice{a&b&p\\c&d&q\\r&s&1}$, la décomposition actuelle est $H=A_1H_{N,\lambda}A_2$ avec
   \[A_1 = \matrice{\frac{\tau\Delta_1-pN+\Gamma_1/N}{\lambda} & \frac{\Delta_1/N}{\lambda} & p-\tau\Delta_1/N\\ \frac{\tau\Delta_2-qN+\Gamma_2/N}{\lambda} & \frac{\Delta_2/N}{\lambda} & q-\tau\Delta_2/N\\ 0 & 0 & 1}, H_{N,\lambda} = \matrice{\lambda & 0 & 0\\0 & \lambda & 0\\N & 0 & 1} \]
   \[\text{et }A_2 = \matrice{r/N & s/N & 0\\-s/N & r/N & \tau\\0 & 0 & 1}\]
   où $N=\sqrt{r^2+s^2}$, $\Delta_1=br-as$, $\Delta_2=dr-cs$, $\Gamma_1 = ar+bs$, $\Gamma_2 = cr+ds$ et $\tau$ et $\lambda$ sont laissés libres.
   
   
  }
 
 \fram{Schema pour le ripmap}{\image{0.5}{pbripmap}}
 
\end{document}
 
 
 
 
 
 
\begin{frame}{Détail de la méthode de Széliski}
\paragraph{Méthode d'interpolation}
\begin{equation*}
img_f(i,j)=\underset{k}{\sum}
\end{equation*}
\end{frame}
